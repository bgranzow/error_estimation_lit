\documentclass{article}
\usepackage{amsmath}
\usepackage[hidelinks]{hyperref}

\title{Error Estimation in Solid Mechanics}
\author{Brian N. Granzow}

\begin{document}

\maketitle

\section{Nomenclature}

\begin{itemize}
\item \emph{Normed error estimate}: Let $u$ denote the exact solution to a
PDE and $u^h$ denote a corresponding finite element approximation.
Error estimates are typically of the form $\| u - u^h \| \leq C(h)$,
where $h$ is some characteristic mesh size and $\| \cdot \|$ is a
chosen global norm.
\item \emph{Functional error estimate}: Functional error estimates
are typically written of the form $| J(u) - J(u^h) | \leq C(h)$,
where $J(u)$ denotes some functional quantity of the solution $u$, which
can potentially be localized. Functional error estimates are a more
general type of error estimate that encompass normed error estimates.
\item \emph{A priori error estimation}: In \emph{A priori} error estimation,
the right hand side $C(h)$ of the estimate depends on $h$ and the
exact solution $u$. Its main use is in qualitative analysis, as the exact
solution $u$ is generally unknown. For instance, an error estimate of the
form:
\begin{gather}
\| u - u^h \| _{L^2} \leq ch^2 | u |_{H^2}
\end{gather}
informs us that a reduction in the mesh size by a factor of 2 will reduce
the error by a factor of 4.
\item \emph{A posteriori error estimation}: In \emph{a posteriori} error
estimation, the right hand side $C(h)$ of the estimate depends on $h$
and the exact solution $u^h$. These types of estimates can be further
manipulated into element-wise bounds, which then qualitatively inform
which elements contribute more greatly to the overall error. This information
can then in turn be used for \emph{adaptive} analysis. For instance,
a simple residual-based \emph{a posteriori} error estimate for Poisson's
equation, $- \nabla^2 u = f$, takes the form:
\begin{gather}
\| u - u^h \| _{L^2} \leq c h \| f + \nabla^2u \|,
\end{gather}
whose right-hand side can be integrated by parts to be expressed as a sum of
volumetric contributions from element interiors and jump contributions from
element boundaries.
\item \emph{Error estimator}: An error estimate $\eta_K$ of the form
$\| u - u^h \| \leq \eta_K$ provides a provably true upper bound for the
discretiztion error that reaches a sharp upper bound as the mesh size
$h \to 0$.
\item \emph{Error indicator}: An error indicator $\hat{\eta}_K$ of the
form $\| u - u^h \| \leq \hat{\eta}_K$ is not a \emph{true} upper bound
on the discretization error. Often times such a true \emph{error estimate}
is difficult to find, for instance, due to nonlinearities in the
underlying PDE. Nonetheless, an indicator approximation $\hat{\eta}_K$
to the true upper bound can be made which still qualitatively behaves
correctly and can be used to inform mesh adaptation. The distinction here
is that the indicator $\hat{\eta}_K$ cannot be used to \emph{guarantee}
that the discretization error is below a certain tolerance.
\end{itemize}

\section{Reviews}

\begin{itemize}
\item \cite{radovitzky1999error}
\item \cite{boussetta2006adaptive}
\item \cite{diez2000adaptivity}
\item \cite{ladeveze2003estimation}
\item \cite{verdugo2014error}
\item \cite{biotteau2012three}
\item \cite{combe2002discretization}
\item \cite{scovazzi2016simple}
\item \cite{molinari2002three}
\item \cite{luo2009adaptive}
\item \cite{lahiri2010variationally}
\item \cite{masud2013framework}
\item \cite{radovitzky1999error}
\item \cite{rodríguez2000error}:
\end{itemize}

\section{Literature Overview}

\section{References}

\bibliographystyle{plain}
\bibliography{references.bib}

\end{document}
