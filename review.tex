\documentclass{article}
\usepackage{amsmath}
\usepackage[hidelinks]{hyperref}

\title{Error Estimation in Solid Mechanics}
\author{Brian N. Granzow}

\begin{document}

\maketitle

\section{Nomenclature}

\begin{itemize}
\item \emph{Normed error estimate}: Let $u$ denote the exact solution to a
PDE and $u^h$ denote a corresponding finite element approximation.
Error estimates are typically of the form $\| u - u^h \| \leq C(h)$,
where $h$ is some characteristic mesh size and $\| \cdot \|$ is a
chosen global norm.
\item \emph{Functional error estimate}: Functional error estimates
are typically written of the form $| J(u) - J(u^h) | \leq C(h)$,
where $J(u)$ denotes some functional quantity of the solution $u$, which
can potentially be localized. Functional error estimates are a more
general type of error estimate that encompass normed error estimates.
\item \emph{A priori error estimation}: In \emph{a priori} error estimation,
the right hand side $C(h)$ of the estimate depends on $h$ and the
exact solution $u$. Its main use is in qualitative analysis, as the exact
solution $u$ is generally unknown. For instance, an error estimate of the
form:
\begin{gather}
\| u - u^h \| _{L^2} \leq ch^2 | u |_{H^2}
\end{gather}
informs us that a reduction in the mesh size by a factor of 2 will reduce
the error by a factor of 4.
\item \emph{A posteriori error estimation}: In \emph{a posteriori} error
estimation, the right hand side $C(h)$ of the estimate depends on $h$
and the exact solution $u^h$. These types of estimates can be further
manipulated into element-wise bounds, which then qualitatively inform
which elements contribute more greatly to the overall error. This information
can then in turn be used for \emph{adaptive} analysis. For instance,
a simple residual-based \emph{a posteriori} error estimate for Poisson's
equation, $- \nabla^2 u = f$, takes the form:
\begin{gather}
\| u - u^h \| _{L^2} \leq c h \| f + \nabla^2u \|,
\end{gather}
whose right-hand side can be integrated by parts to be expressed as a sum of
volumetric contributions from element interiors and jump contributions from
element boundaries.
\item \emph{Error estimator}: An error estimate $\eta_K$ of the form
$\| u - u^h \| \leq \eta_K$ provides a provably true upper bound for the
discretiztion error that reaches a sharp upper bound as the mesh size
$h \to 0$.
\item \emph{Error indicator}: An error indicator $\hat{\eta}_K$ of the
form $\| u - u^h \| \leq \hat{\eta}_K$ is not a \emph{true} upper bound
on the discretization error. Often times such a true \emph{error estimate}
is difficult to find, for instance, due to nonlinearities in the
underlying PDE. Nonetheless, an indicator approximation $\hat{\eta}_K$
to the true upper bound can be made which still qualitatively behaves
correctly and can be used to inform mesh adaptation. The distinction here
is that the indicator $\hat{\eta}_K$ cannot be used to \emph{guarantee}
that the discretization error is below a certain tolerance.
\item \emph{Correction indicator}: A  \emph{correction indicator}
$\tilde{\eta}_K$ which is used to drive mesh adaptation at the element
level and may or may not have any mathematical basis and/or actual relation
to the discretization error itself. For example a heuristic correction
indicator might indicate refinement at the interface between two materials.
\end{itemize}

\section{Reviews}

\begin{itemize}
\item
\emph{Error estimation and adaptive meshing in strongly nonlinear dynamic
problems} \cite{radovitzky1999error}: The authors develop a framework for
estimating errors in nonlinear solid mechanics problems that are potentially
dynamic. They assume that the material obeys a minimum principle, valid
for a range of material models with an appropriate constitutive update,
and use this minimum principle as the basis for error estimation. They
discretization the PDE in time first and then discretize in space leading
to an equivalent static problem on which error estimation is performed.
They show results for error estimation and mesh adaptation for compressive
waves traveling down a shock tube in three dimensions with an internal
temperature state-variable.
\item
\emph{Adaptive remeshing based on a posteriori error estimation for forging
simulation} \cite{boussetta2006adaptive}: The authors present an approach
mesh adaptation in forging simulations with a viscoplastic material model
using ZZ/SPR-type error estimation. The governing equations do not include
inertial terms.
\item
\emph{Adaptivity based on error estimation for viscoplastic softening
materials}  \cite{diez2000adaptivity}: The authors develop an adaptive
approach for viscoplastic materials by generalizing a residual-type
error estimation procedure. This residual error estimate is split into two
parts: local solves over element interiors and then by solving Dirichlet
problems over patches of elements. Results for quasi-steady problems in
two dimensions with strain-localization are considered.
\item
\emph{Estimation of discretization errors in dynamics}
\cite{ladeveze2003estimation}: The authors develop error estimates for
transient dynamics problems using the so-called constitutive relation error,
where the inertial term $\rho \ddot{u}$ is treated as a constitutive
relationship. They develop both spatial and temporal error indicators
and show results for simple linear elastic problems.
\item
\emph{Error assessment in structural transient dynamics}
\cite{verdugo2014error}: The authors present a review of current
state-of-the-art techniques for linear elastodynamics. They break
estimation techniques into four categories: recovery-based, dual-weighted
residual or adjoint-based, constitutive relation error, and
timeline-dependent functional error estimates. They review each method
and discuss approximating both time and spatial discretization errors.
\item
\emph{Three dimensional automatic refinement method for transient small
strain elastoplastic finite element computations} \cite{biotteau2012three}:
The authors present an adaptive procedure for dynamic elastoplasticity.
They use a geometric multigrid procedure for the solution of the
model problem and leverage the fact that the solution exists at different
multigrid levels to develop three different types of indicators
for the spatial discretization error.
\item
\emph{Discretization error estimator for transcient dynamic simulations}
\cite{combe2002discretization}: This is literally almost the same
paper as \cite{ladeveze2003estimation}.
\item
\emph{A simple, stable, and accurate linear tetrahedral finite element for
transient, nearly, and fully incompressible solid dynamics: a dynamic
variational multiscale approach} \cite{scovazzi2016simple}: The
authors introduce an extended VMS stabilized tetrahedral element for
transient dynamics, where transient effects are considered in addition
to standard VMS techniques. While they do not explore error estimation,
the VMS method comes naturally equipped with a point-wise error estimate,
and studying error estimation with this type of approach could be
an interesting avenue for further research.
\item
\emph{Three-dimensional adaptive meshing by subdivision and edge-collapse in
finite-deformation dynamic plasticity problems with application to adiabatic
shear banding} \cite{molinari2002three}: This is an extension of the work
\cite{radovitzky1999error} to plasticity formulations. The authors perform
mesh adaptation (refinement and coarsening) for an adiabatic shear-banding
example based on \emph{a posteriori} error estimates. Their approach
appears to require elements of greater order than 1.
\item
\emph{Adaptive nearest-nodes finite element method guided by gradient of
linear strain energy density} \cite{luo2009adaptive}: The authors use
the gradient of the strain energy density to guide mesh modification.
That is, in areas where there is a large strain energy density, they
create a denser mesh and in areas where there there is a small strain
energy density, there is less mesh resolution. The relative error in the
total potential energy between adaptive iterations is used as a stopping
criteria for the adaptive process. They apply their approach to
2D problems in static elasticity.
\item
\emph{A variationally consistent mesh adaptation method for triangular
elements in explicit Lagrangian dynamics} \cite{lahiri2010variationally}:
The authors apply a ZZ-type error estimation procedure to recover
`improved` stress values which serve as the basis for error estimation.
Their main motivation is to describe mesh modification procedures that ensure
the conservation of linear and angular momentum in explicit Lagrangian
dynamics. They apply their adaptive approach to some 2D problems.
\item
\emph{A framework for residual-based stabilization of incompressible finite
elasticity: Stabilized formulations and methods for linear triangles and
tetrahedra} \cite{masud2013framework}: The authors develop a VMS approach
for triangular and tetrahedral elements with equal-order pressure-displacement
elements. They use the natural properties of the VMS method (a so-called
fine-scale solution) as the basis for error estimation and investigate
error estimates for a variety of quasi-steady problems in finite elasticity.
They show at a minimum that the error is qualitatively correctly
distributed.
\item
\emph{Variational h-adaption in finite deformation elasticity and plasticity}
\cite{rodríguez2000error}: This follows the same principles as papers
\cite{radovitzky1999error} and \cite{molinari2002three}. They however only
consider quasi-steady problems in this paper.  \item
\emph{Error estimation and adaptivity for nonlocal damage models}
\cite{rodríguez2000error}: The authors perform mesh adaptation for damage
models using a residual-type error estimate very similar to the one in
reference \cite{diez2000adaptivity}. They apply this approach to some pretty
cool quasi-steady problems in two dimensions.
\end{itemize}

\section{Literature Overview}

The history of \emph{a posteriori} error estimation and mesh adaptation is
intimately linked with solid mechanics applications. Despite this rich
history, \emph{a posteriori} error estimation has remained relatively
unexplored for (a) dynamic (b) nonlinear solid mechanics problems with
(c) history-dependent variables. In this context, Ortiz et al.
\cite{radovitzky1999error, mosler2007variational} studied \emph{a posteriori}
error estimation in a global norm based on a principle of minimum potential
energy for adiabatic shear-banding problems and Biotteau et al.
\cite{biotteau2012three} have studied \emph{a posteriori} error estimation in
the energy norm for dynamic elasto-plasticity using a multi-grid approach.

It is, however, more common to find methods developed for some subset of
the properties (a)-(b). For instance, for nonlinear history-dependent
mechanics applications,  Boussetta et al. \cite{boussetta2006adaptive}
utilized a Zienkiwicz-Zhu type error indicator to perform adaptive mesh
refinement in metal forming simulations, Mosler and Ortiz et al. utilized the
same error estimation approach as Ortiz et al. \cite{radovitzky1999error,
mosler2007variational} to study quasi-static elasto-plasticity problems,
and D{\'i}ez et al. \cite{diez2000adaptivity} and Rodr{\'i}guez et al.
\cite{rodríguez2000error} used a residual-based error estimation technique
to study problems in viscoplasticity and non-local damage models,
respectively.

In the context of truly transient solid mechanics problems that include
inertial terms, Ladev{\'e}ze et al. \cite{ladeveze2003estimation,
combe2002discretization} for linear elastodynamics problems using the
so-called constitutive-relation error. Lahiri et al. utilized a
Zienkiwicz-Zhu type error indicator for the Cauchy stress tensor to perform
mesh adaptation for explicit Lagrangian dynamics. Lastly,
\cite{verdugo2014error} presents a comprehensive review of current
state-of-the art methods for \emph{a posteriori} error estimation in
elastodynamics in both global norms and in functional quantities of interest,
as discussed in the next paragraph.

Rather than estimating errors in a global norm, it is also possible to
estimate the discretization error in physically meaningful functional
quantities of interest using adjoint-based techniques. 
In the context of solid mechanics, adaptive adjoint-based error estimation
has been used to study linear elasticity in two
\cite{rannacher1997feed, stein2007error, gonzalez2014mesh} and three
\cite{ghorashi2014goal} dimensions, two
\cite{rannacher1998posteriori, rannacher1999posteriori} and three
\cite{ghorashi2017goal} dimensional elasto-plasticity, two dimensional
thermoelasticity \cite{rabizadeh2015adaptive}, two dimensional
nonlinear elasticity \cite{larsson2002strategies}, two dimensional
finite elasticity \cite{whiteley2014error}, and three dimensional
finite elasticity \cite{granzow2018adjoint}. We remark that all of these
studies assume quasi-steady governing equations, with the exception
of \cite{verdugo2014error}.

Finally, we remark that there has been recent interest in developing
Variational Multiscale (VMS) methods for solid mechanics applications
\cite{masud2013framework, masud2013framework, scovazzi2016simple}.
The VMS method is naturally equipped with a point-wise error estimate
that can be used to derive global error estimates.  Masud and Truster
investigated the performance of this estimate in the context of
finite deformation elasticity and Scovazzi et al. developed an extended
VMS method for truly transient elastic behavior. The work of Scovazzi
could potentially be extended to include error estimation results.

I'll remark here that all of these works involve simplical elements
except for reference \cite{biotteau2012three}.

\bibliographystyle{plain}
\bibliography{references.bib}

\end{document}
